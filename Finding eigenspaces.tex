\documentclass[reqno,a4paper,12pt]{amsart}
\usepackage{amsmath}
\usepackage{setspace}
\usepackage{amssymb}

\newcommand{\R}{\mathbb{R}}

\author{Lewis McConkey}
\date{\today}
\title{ \LARGE{Math220/Math240} \\
   \setstretch{1}Individual Coursework}


\begin{document}
\maketitle
 
 \begin{align}
    A 
    =
    \begin{bmatrix}
        2  &  1  &  -1  \\
        0  &  1  &   1  \\
        0  &  0  &  2
    \end{bmatrix} \label{eq:1}
  \end{align}\\
Since the matrix A is upper triangular the eigenvalues are the entries of the diagonals. The characteristic polynomial is $(2-\lambda)^2(1-\lambda)$. Hence the eigenvalues are $\lambda = 1,  \lambda = 2$ (repeated).\\
\\In the case when $\lambda = 1$:\\
Finding all vectors such that $(A- I_{3}) \overrightarrow{x} = \overrightarrow{0}$. In other words:\\

  \begin{align}
    \begin{bmatrix}
       1  &  1  &  -1 \\
       0  &  0  &   1 \\ 
       0  &  0  &   1
    \end{bmatrix}
    \begin{bmatrix}
      x \\
      y \\
      z 
    \end{bmatrix}
    =
    \begin{bmatrix}
    0 \\
    0 \\
    0 \\
    \end{bmatrix} \label{eq:2}
  \end{align}\\
From equation \eqref{eq:2} we get $x+y-z=0, y=0, z=0$. So $x=y=z=0$. Hence the solution set to this system of equations is the eigenspace:\\ 
  \begin{align}
  V_1
  =
  \{
    \begin{bmatrix}
      x \\
      y \\
      z \\
    \end{bmatrix}
    |
    \hspace{0.2cm}x=y=z=0
  \}
  = 
  span_\R
  \{
  \begin{bmatrix}
  0 \\
  0 \\
  0 \\
  \end{bmatrix}
  \}\label{eq:3}
  \end{align}\\
In the case when $\lambda = 2$:\\
Finding all vectors such that $(A- 2I_{3}) \overrightarrow{x} = \overrightarrow{0}$. In other words:\\

  \begin{align}
    \begin{bmatrix}
       0  &  1  &  -1 \\
       0  & -1  &   1 \\ 
       0  &  0  &   0
    \end{bmatrix}
    \begin{bmatrix}
      x \\
      y \\
      z 
    \end{bmatrix}
    =
    \begin{bmatrix}
    0 \\
    0 \\
    0 \\
    \end{bmatrix}\label{eq:4}
  \end{align}\\
     
From equation \eqref{eq:4} we get $x-y=0, -y+z=0, 0=0$. So $x=y=z$. Hence the solution set to this system of equations is the eigenspace\\ 
  \begin{align}
  V_2
  =
  \{
    \begin{bmatrix}
      x \\
      y \\
      z \\
    \end{bmatrix}
    |
    \hspace{0.2cm}x\in\R
  \}
  = 
  span_\R
  \{
  \begin{bmatrix}
  1 \\
  1 \\
  1 \\
  \end{bmatrix}
  \}\label{eq:5}
  \end{align}\\
The two vectors (1,1,1) and (0,0,0) are both eigenvectors. However they are not linearly independent so by theorem 1.44. they don't form a basis of $\R^3$.

     
\end{document}