\documentclass[reqno,a4paper,12pt]{amsart}
\usepackage{amsmath}
\usepackage{setspace}
\usepackage{amssymb}
\usepackage{mathtools}
\usepackage{mathrsfs}

\newcommand{\R}{\mathbb{R}}
\newcommand{\C}{\mathscr{C}}
\newcommand{\B}{\mathscr{B}}

\numberwithin{equation}{section}

\author{Lewis McConkey}
\date{\today}
{\title{ \LARGE{Math220/Math240 \\
          \hspace{16cm}Individual Coursework}}}
   
   
   
\begin{document}
\maketitle
\section{Computing $^\C[T]^\C$,$^\B[T]^\C$,$^\C[T]^\B$ and $^\B[T]^\B$}

For each of the basis vectors in $\C$ we apply T, and the coordinates of those vectors with respect to $\C$ will be the columns of $^\C[T]^\C$

  \begin{align}
  T(
    \begin{pmatrix}
      1 , 0 \\
    \end{pmatrix}
    )
    =
    (2,1),\hspace{0.2cm}
    T(
    \begin{pmatrix}
    0 , 1 \\
    \end{pmatrix}
    )
    =
    (-1,3) \label{eq:1.1}
  \end{align}\\
Since $\C$ is the standard basis of $\R^2$ the resulting matrix is as follows:
  \begin{align}
     ^\C[T]^\C 
     =
     \begin{bmatrix}
       2 & -1 \\
       1  & 3
     \end{bmatrix} \label{eq:1.2}
   \end{align}\\
Now finding $^\B[T]^\C$. We already know what we get by applying T to the $\C$ basis vectors by \eqref{eq:1.1}. Notice that:

  \begin{align}
    \begin{pmatrix}
      2 , 1 \\
    \end{pmatrix}
    =
    \begin{pmatrix}
    1,0
    \end{pmatrix}
    +
    \begin{pmatrix}
    1,1
    \end{pmatrix},\hspace{0.2cm}
    \begin{pmatrix}
      -1 , 3 \\
    \end{pmatrix}
    = 
    -4
    \begin{pmatrix}
     1,0
    \end{pmatrix}
    +
    3
    \begin{pmatrix}
    1,1
    \end{pmatrix} \label{eq:1.3}
  \end{align}\\
Hence the matrix is as follows:
  \begin{align}
     ^\B[T]^\C 
     =
     \begin{bmatrix}
       1 & -4 \\
       1 & 3
     \end{bmatrix} \label{eq:1.4}
   \end{align}\\
For each of the basis vectors in $\B$ we apply T, and the coordinates of those vectors with respect to $\C$ will be the columns of $^\C[T]^\B$

  \begin{align}
  T(
    \begin{pmatrix}
      1 , 0 \\
    \end{pmatrix}
    )
    =
    (2,1),\hspace{0.2cm}
    T(
    \begin{pmatrix}
    1 , 1 \\
    \end{pmatrix}
    )
    =
    (1,4) \label{eq:1.5}
  \end{align}\\
Since $\C$ is the standard basis of $\R^2$ the resulting matrix is as follows:
  \begin{align}
     ^\C[T]^\B 
     =
     \begin{bmatrix}
       2 & 1 \\
       1  & 4
     \end{bmatrix} \label{eq:1.6}
   \end{align}\\
Now finding $^\B[T]^\B$. We already know what we get by applying T to the $\B$ basis vectors by \eqref{eq:1.5}. Notice that:

  \begin{align}
    \begin{pmatrix}
      2 , 1 \\
    \end{pmatrix}
    =
    1
    \begin{pmatrix}
    1,0
    \end{pmatrix}
    +
    1
    \begin{pmatrix}
    1,1
    \end{pmatrix},\hspace{0.2cm}
    \begin{pmatrix}
      1, 4 \\
    \end{pmatrix}
    = 
    -3
    \begin{pmatrix}
     1,0
    \end{pmatrix}
    +
    4
    \begin{pmatrix}
    1,1
    \end{pmatrix} \label{eq:1.7}
  \end{align}\\
Hence the matrix is as follows:
  \begin{align}
     ^\B[T]^\B 
     =
     \begin{bmatrix}
       1 & -3 \\
       1 & 4
     \end{bmatrix} \label{eq:1.8}
   \end{align}\\   
   
\section{Computing $^\B[T \circ T]^\B$}
We need to apply T twice to the $\B$ basis vectors and find the coordinates of these with respect to $\B$. These coordinates will be the columns of the matrix $^\B[T \circ T]^\B$\\
  \begin{align}
    T(
    T(
    \begin{pmatrix}
    1,0
    \end{pmatrix}
    )
    =
    T(
    \begin{pmatrix}
    2,1
    \end{pmatrix}
    )
    =
    \begin{pmatrix}
    3,5
    \end{pmatrix}
    =
    -2
    \begin{pmatrix} 
    1, 0
    \end{pmatrix}
    +
    5
    \begin{pmatrix}
    1, 1
    \end{pmatrix} \label{eq:2.1}\\
    T(
    T(
    \begin{pmatrix}
    1,1
    \end{pmatrix}
    )
    =
    T(
    \begin{pmatrix}
    1,4
    \end{pmatrix}
    )
    =
    \begin{pmatrix}
    -2,13
    \end{pmatrix}
    =
    -15
    \begin{pmatrix} 
    1, 0
    \end{pmatrix}
    +
    13
    \begin{pmatrix}
    1, 1
    \end{pmatrix} \label{eq: 2.2}
  \end{align}\\
Hence the matrix is as follows:\\
  \begin{align}
    ^\B[T \circ T]^\B
    =
    \begin{bmatrix}
    -2  &  -15 \\
    5  &  13
    \end{bmatrix} \label{eq: 2.3}
  \end{align}\\
  
  \section{Verifying that ($^\B[T]^\C$)($^\C[T]^\B$) = $^\B[T \circ T]^\B$ = ($^\B[T]^\C$)($^\C[T]^\B$)}
  \begin{align}
  (^\B[T]^\C)(^\C[T]^\B) 
  =
    \begin{bmatrix}
      1 & -4 \\
      1 & 3
    \end{bmatrix}
    \begin{bmatrix}
      2 & 1 \\
      1 & 4
    \end{bmatrix}
    =
    \begin{bmatrix}
      -2 & -15 \\
      5 & 13
    \end{bmatrix}
    =
    \hspace{0.05cm}^\B[T \circ T]^\B \label{eq:3.1}
  \end{align} 
\begin{align}
(^\B[T]^\B)(^\B[T]^\B) 
  =
    \begin{bmatrix}
      1 & -3 \\
      1 & 4
    \end{bmatrix}
    \begin{bmatrix}
      1 & -3 \\
      1 & 4
    \end{bmatrix}
    =
    \begin{bmatrix}
      -2 & -15 \\
      5 & 13
    \end{bmatrix}
    =
    \hspace{0.05cm}^\B[T \circ T]^\B \label{eq:3.1}
  \end{align} 
These two equations hold true by equation \eqref{eq: 2.3}. Hence:\\
\begin{align}
(^\B[T]^\C)(^\C[T]^\B) = ^\B[T \circ T]^\B = (^\B[T]^\C)(^\C[T]^\B)
\end{align}
  
\end{document}