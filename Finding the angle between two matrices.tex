\documentclass[reqno,a4paper,12pt]{amsart}
\usepackage{amsmath}
\usepackage{amssymb}

\author{Lewis McConkey}
\date{\today}
\title{ \LARGE{Finding the angle} \\
   \hspace{16cm}between two matrices}


\begin{document}
\maketitle
Let:
 \begin{align}
    A 
    =
    \begin{bmatrix}
        1  &  2 &  0  \\
        2  &  1  &   -2  \\
        0  &  -2  &  1
    \end{bmatrix}\hspace{0.2cm}
    B
    =
    \begin{bmatrix}
        1 &  0 &  0  \\
        0  &  1  &   0  \\
        0  &  0  &  8
    \end{bmatrix} \label{eq:1}
  \end{align}\\
First we need to calculate $\langle A,B \rangle$ with respect to the inner product:
    \begin{align}
      \langle A,B \rangle
      := 
      tr(AB^T) \label{eq:2}
     \end{align}\\
In the case where A and B are the matrices stated above:
     \begin{align}
      tr(
      \begin{bmatrix}
        1  &  2 &  0  \\
        2  &  1  &   -2  \\
        0  &  -2  &  1
    \end{bmatrix} 
    \begin{bmatrix}
        1 &  0 &  0  \\
        0  &  1  &   0  \\
        0  &  0  &  8
    \end{bmatrix} 
      ) 
      =
      tr(
      \begin{bmatrix}
        1 & 2 & 0 \\
        2 & 1 & -16 \\
        0 & -2 & 8
      \end{bmatrix}
      )
      =
      10 \label{eq:3}
    \end{align}\\
 Next we need to calculate $\|A\|$ and $\|B\|$:
    \begin{align}
      \|A\|
      =
      \langle A,A \rangle
      =
      tr(AA^T)\hspace{0.5cm}
      \|B\|
      =
      \langle B,B \rangle
      =
      tr(BB^T) \label{eq:4}
    \end{align}\\
Computing $tr(AA^T)$ and $tr(BB^T)$ in the same way as we did in \eqref{eq:2} and in \eqref{eq:3} by multiplying A by A transpose and B by B transpose and finding the trace of the resulting matrices we get that:
    \begin{align}
      \|A\|
      =
      \sqrt{19}\hspace{0.5cm}
      \|B\|
      =
      \sqrt{66}  \label{eq:5}
    \end{align}\\
Therefore inputting our findings from \eqref{eq:3} and \eqref{eq:5} into the formula given in the lecture notes we get that:
    \begin{align}
      \cos\theta = 
      \frac{\langle A,B \rangle}{\|A\|\|B\|}
      =
      \frac{10}{\sqrt{19} \cdot \sqrt{66}} \label{eq:6}
    \end{align}
Hence using a calculator we get that $\theta= 73.597$ (3 d.p.)

     
\end{document}