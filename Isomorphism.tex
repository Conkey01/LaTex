\documentclass[,oneside]{article}
\usepackage{amsmath}
\usepackage{amsthm}
\usepackage{setspace}
\usepackage{amssymb}
\usepackage{enumerate}
\usepackage{enumitem}
\usepackage{hyperref}
\usepackage{graphicx}
\numberwithin{equation}{section}
\usepackage{listings}
\newcommand{\Z}{\mathbb{Z}}

\author{Lewis McConkey}
\date{\today}
\title{ \LARGE{Isomorphism and Homomorphism}}
     
          
\begin{document}
\maketitle
\section{Prerequisite}
The following document is based on these two questions from a module on Abstract Algebra:
\begin{enumerate}[i]
\item Let G and H be groups, and suppose $\theta : G \rightarrow H$ is an isomorphism (in the sense of
Definition 6.9). Prove that if G is Abelian, then H is Abelian
\item Let V be “the” group from Section 4.1, i.e. a group of order 4 in which each non-identity element has order 2. Let $\theta : V \rightarrow \Z_4$ be a bijection. Without using any Cayley tables, show that $\theta$ cannot be a group homomorphism.
\end{enumerate}

\section{Isomorphism}
For the first proof will we use the following definition of isomorphism:\\
\begin{center}
\begin{definition}{\label=\textbf{Definition 6.9}}
Let $(G, *_G)$ and $(H, *_H )$ be groups. We say that G and H are isomorphic (as groups), denoted by $G \cong H$, if there is a bijection $\phi : G \rightarrow H$ satisfying
$\phi (g_1 *_G g_2) = \phi (g_1) *_H \phi (g_2) \hspace{0.1cm}\forall g_1, g_2 \in G.$
\end{definition}
\end{center}
\begin{proof}
Suppose G is Abelian, now we have $xy=yx \hspace{0.1cm}\forall x,y \in G$.\\
We now consider elements $c, d \in H$, since $\phi$ is an isomorphism from definition 6.9 we see that we have $c=\theta(a)$ and $d=\theta(b)$ for $a,b \in G$. Hence, 
\begin{center}
$cd=\phi(a) \phi(b)= \phi(ab)= \phi(ba)=\phi(b)\phi(a)=dc$
\end{center}
Notice how $\phi(ab)=\phi(ba)$ since G is supposed to be Abelian.\\
This proves that cd=dc for elements $c,d \in H$ therefore H is also Abelian 
\end{proof}

\section{Homomorphism}
For the second question we will set out to prove against one of the properties of homomorphism which is:
\begin{center}
Let $\phi: G \rightarrow$ H be a group homomorphism.\\
(iv) $o(\phi(g))$ divides $o(g) \hspace{0.1cm}\forall g \in G.$
\end{center}
\begin{proof}
Consider $g \in G$ to be a non-identity element so that now $o(g)=2$\\
Now we need to prove that $o(\theta(g)) \nmid 2$:\\
Take $g \in G$ to be the element that maps to $\hat{2} \in \Z_4$:\\
Hence $\theta(g)=\hat{2}$
\begin{center}
$\hat{2} \cdot \hat{2} = \hat{4} = \hat{1}$ \\
$\hat{2} \cdot \hat{2} \cdot \hat{2} = \hat{1} \cdot \hat{2} = \hat{2}$
\end{center}
This shows that $o(\theta(g)) = o(\hat{2}) = 3$\\
Therefore $\theta$ cannot be a group homomorphism since $3 \nmid 2$. Hence we have proved against (iv) because $o(\theta(g)) \nmid o(g) \hspace{0.1cm} \forall g \in G$
\end{proof}


\end{document}