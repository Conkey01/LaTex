\documentclass[,oneside]{article}
\usepackage{amsmath}
\usepackage{amsthm}
\usepackage{setspace}
\usepackage{amssymb}
\usepackage{enumerate}
\usepackage{enumitem}
\usepackage{hyperref}
\usepackage{graphicx}
\numberwithin{equation}{section}
\usepackage{listings}
\newcommand{\Z}{\mathbb{Z}}

\author{Lewis McConkey}
\date{\today}
\title{ \LARGE{Proof that $\sqrt7$ is irrational}}
     
          
\begin{document}
\maketitle
\begin{proof}
We will set out to prove this by contradiction, assuming that $\sqrt7$ is rational. By definition of rational numbers:
\begin{center}
$\sqrt7=\frac{a}{b}$ for $a,b \in \Z$ and b $\neq$ 0\\
$7=\left( \frac{a}{b} \right)^{2}$ (squaring both sides)\\
$7b^2=a^2$ (multiply by $b^2$ on both sides)\\
\end{center}
Therefore $a^2$ is a multiple of 7 and a must also be a multiple of 7\\
Now let's write $a=7k$ for $k \in \Z$ and substituting $a=7k$ into the equation $7b^2=a^2$ we get: 
\begin{center}
$7b^2=7k^2$\\
&7b^2=49k^2&\\
&b^2=7k^2&\\
\end{center}
Therefore $b^2$ is also a multiple of 7 and so is b. However this contradicts the assumption that a and b have no common factors. Therefore our assumption that $\sqrt7$ is rational leads to a contradiction. Therefore we conclude that $\sqrt7$ is irrational. 
\end{proof}


\end{document}